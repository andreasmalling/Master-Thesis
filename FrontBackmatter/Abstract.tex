%*******************************************************
% Abstract
%*******************************************************
%\renewcommand{\abstractname}{Abstract}
\pdfbookmark[1]{Abstract}{Abstract}
\begingroup
\let\clearpage\relax
\let\cleardoublepage\relax
\let\cleardoublepage\relax

\chapter*{Abstract}
The purpose of this study is to determine the suitability of using \acs{IPFS} as a way to stream videos in a distributed network, with the purpose of offloading  high sudden demand that can occur on viral videos.

The study is performed by simulating users with different behaviours and network conditions, in a Docker environment, and having them stream videos in a real \acs{DASH} video player while using \acs{IPFS} to retrieve the relevant video files. Results are obtained by having the users report metrics regarding viewing experience and network performance.

A large weakness of \acs{IPFS} was discovered by the experiments, as the largest amount of data received by each user was duplicate data with a correlation to the number of seeding peers on said data. However the amount of duplicate data fell when sharing larger file segments.

In terms of users, the impact of their behaviour was mostly isolated within themselves, and leeching was punished by a bad user experience. Performance did however decrease for everyone when the amount of leeching users grew large enough.

\acs{IPFS} was found to have a very high \acs{CPU} usage, meaning the amount of users that could be simulated was severely limited.

The results show that \acs{IPFS} is ill-suited for streaming in its current state, as this involves sharing many small files which yields a large overhead of unnecessary data and network traffic. The amount of duplicate data also made \acs{IPFS} a very poor choice for mobile users as their limited bandwidth would be used on duplicates instead of relevant data.

\vfill

\endgroup			

\vfill