\chapter{Introduction}
\label{cha:introduction}
Videos require a lot of space, and with an ever going stream of new content, the videos, hosted on sites such as YouTube, only generate a large amount of traffic for a short time. Resulting in a potential high amount of access in a short time span, exemplified in viral videos. 

This means that a high hosting capacity is needed in both space and bandwidth, to meet potential demand and avoid a flooding in page requests that exceeds the resource's available bandwidth or the ability of its servers to respond, and render the resource temporarily unreachable.

This is a somewhat common problem, and is often associated with social platforms such as \textit{Twitter} or \textit{Reddit}, where the latter have named the effect a "hug of death", which is a familiar term on the website.

\vspace{0.8cm}
\noindent This report will examine the following hypothesis:

\begin{displayquote}
    \textit{
        By utilizing \ac{IPFS} as a back-end for a \ac{DASH} video-sharing website, videos are expected to be available in their entirety, viewable without stalling or re-buffering and being persistent, though no adaptive bit rate is utilized.
        This should be the case for any media and especially apply to circumstances of surging network traffic due to high demand. The increased consumption will allow an increase in availability due to seeding from consumers of the video.
    }
\end{displayquote}
\vspace{0.5cm}

By utilizing the \ac{IPFS} for hosting videos, a server-less video hosting website will be implemented.

The \ac{DASH} technology will be tested in simulation of different demands, with different segments of the videos watched and different bitrates for the videos. This will be done without multiple bitrate sources, forcing IPFS to only download from a single stream of video and audio, making it more available to the \ac{P2P} network, and easier to interpret.

The feasibility of the system will be assessed by the \ac{UX} in the experiments, where as a measurement for \ac{UX} the focus lies on the viewing experience of the video such as the number of interruptions for needed buffering and the duration of such. This will be done by a number of \emph{Personas} with different use behaviors, to assess the impact a client have on the entire network and others \ac{UX}.


The network setup will be exposed to various network obstacles, such as low bandwidth, high latency and others. This will all be done with inspiration from Chaos Engineering, by creating a hypothesis, setting up a \emph{stable} network and introduce the variables for the experiment, to either confirm or dismiss said hypothesis.


It is expected that popular content will perform better than lesser known, and thereby lesser watched videos, which should make it suitable to avoid unreachability when a sudden surge of traffic occurs.

% The results will be theoretically compared with other video streaming systems, centralized as well as distributed eg., with BitTorent as a back-end.

%%% Local Variables:
%%% mode: latex
%%% TeX-master: "../ClassicThesis"
%%% ispell-dictionary: "british" ***
%%% mode:flyspell ***
%%% mode:auto-fill ***
%%% fill-column: 76 ***
%%% End:
