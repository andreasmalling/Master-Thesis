\chapter{Introduction}
\label{cha:introduction}
Videos require a lot of space, and with an evergoing stream off news, the videos hosted on sites such as YouTube remain relevant in only short amounts of time, resulting in a potential high amount of access in a short amount of time, exemplified in viral videos.

This means that a high hosting capacity is needed in both space and bandwidth.

This report will try to answer the following hypothesis:

\begin{displayquote}
    \textit{
        Can the \acl{IPFS} be utilized as a feasible back-end for the \acl{DASH} while still delivering a good \acl{UX}. \\ \\
        As a measurement for \acl{UX} videos are expected to be available in their entirety, viewable without stutter and being persistent. It is expected that popular content will perform better than lesser known, and thereby watched, videos.
    }
\end{displayquote}

By utilizing the \ac{IPFS} as a base for hosting videos, a server-less video hosting website will be implemented.

The \ac{DASH} will be tested in simulation of different demands, different segments of the videos watched and different sizes/qualities of the videos.

The results will be theoretically compared with other video streaming systems, centralized as well as distributed (eg. with BitTorent as a back-end).

%%% Local Variables:
%%% mode: latex
%%% TeX-master: "../ClassicThesis"
%%% ispell-dictionary: "british" ***
%%% mode:flyspell ***
%%% mode:auto-fill ***
%%% fill-column: 76 ***
%%% End:
