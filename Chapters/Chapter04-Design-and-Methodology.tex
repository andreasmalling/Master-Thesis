\chapter{Design and Methodology}
\label{cha:design-and-method}
\section{Video content and encoding}
Splitting up videos allows for easier \acs{IPFS} sharing. This is possible due to the DASH live profile, with supports multiple files (segments) to form a stream. As a consequence each file should begin with an I-Frames.

Separate audio video streams (no multiplexing) is needed, but only to be DASH-IF compliant making it possible to use the DASH.JS player.

\section{Personas}
Possible user types could include:
\begin{itemize}
    \item \textbf{Leecher:}
    Not sharing videos back
    \item \textbf{Mobile User:}
    Unreliable, fluctuating network connection
    \item \textbf{Good User:}
    User sharing content, either origin or re-hosting
\end{itemize}

\section{Evaluation Experiments}
Experiments for measuring \ac{QOE}, meaning no re-buffering and stalls of streams;
\begin{itemize}
    \item \textbf{Bandwidth:}
    Both the upload and download speeds of the client can be manipulated, and at which point does these become to low to get a pleasant viewing experience, meaning that the video does not halt after it has initially started. Also how does the low upload affect the other peers, and what innfluence does fluctuating bandwidth capacity have such as that of a mobile client.
    \textit{Cases of eg. mobile devices.}
    \item \textbf{Churnrate:}
    What effect does clients leaving the network have on the video, and can videos partially or entirely disappear from the network, can a video be popular enough that this cannot reasonably occur.
    \item \textbf{Segments availability:}
    What impact does the popularity of the video have in terms of availability
    \textit{Cases of spike in users, unpopular video performance. }
    \item \textbf{Video Quality:}
    Influence of resolution, \acs{FPS}, I-Frame interval, bit rate as in \cite{aloman2015performance}.
    \item \textbf{Buffersize:}
    Influence of changing DASH preferred buffersize. Done dynamically already, so maybe not interesting. 
    \item \textbf{Public IPFS:}
    How does IPFS perform in a small environment dedicated only to running the tests versus running the test in the global IPFS network that is also used for other unrelated services.
    \textit{Case of "in the wild" performance.} 
\end{itemize}
%%% Local Variables:
%%% mode: latex
%%% TeX-master: "../ClassicThesis"
%%% End:
