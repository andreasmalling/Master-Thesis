\chapter{Analysis}
\label{cha:analysis}

\ac{IPFS} is an unexplored technology for the purpose of video streaming. Many have built video streaming services based on \ac{P2P} technologies such as BitTorrent and \ac{DHT}s \cite{gazdar2017toward}.

A common theme is that the videos are split into smaller segments, which can be distributed between peers. An assumption none of the referenced work has made is that some pieces can become rare because they are simply skipped. For example, the last segments of a video could not have been buffered if the user closed the stream prematurely. How the user interacts with the system can vary greatly and is unexplored by other works, therefore it is one of the major focus points of our testing.

\section{Evaluation method}
Inspired by the following articles from \autoref{cha:related-work}, an evaluation strategy is formed.
\citeauthor{aloman2015performance} \cite{aloman2015performance} had multiple parameters they could adjust on the videos, such as bitrate, interval between key frames and more, and then evaluated different systems based on user experience. But their evaluation only accounted for a single type of user. Their testing was also based on a client-server model rather than \ac{P2P}, which might impact the performance of the \ac{DASH} protocol. But according to \citeauthor{nguyen2009p2p} a \ac{P2P} system would reduce the bandwidth, resulting in better user experience. Their testing focused on probability of retrieving video pieces and latency required to do so. However \ac{IPFS} does not use a \ac{RNC} so their results might not be entirely relatable unlike their testing parameters.

\citeauthor{gazdar2017toward} \cite{gazdar2017toward} did make a \ac{P2P} \ac{DASH} video streaming service unlike \citeauthor{aloman2015performance}, but rather than making it based on \ac{IPFS}, they based it on BitTorrent. Their testing was again focused on user experience, while adjusting the network conditions. In their evaluation however they also only had a single type of user. Their results contradicted expectation as they state that they get worse results as the  network grew. \citeauthor{nguyen2009p2p} mentions that multiple internet routes should give reduced bandwidth. So it is likely that their peer selection algorithm did not try to use this to their advantage.

These different evaluations were all focused on evaluating the performance  and the \acp{UX} based on different conditions such as video quality metrics and network conditions. However, none of them went the other way to see how the users behaviours can affect the network and even their own \ac{UX}.

\citeauthor{broxton2013catching} \cite{broxton2013catching} defined socialness for videos, but did not build a system to see how socialness of a video can affect the user experience, and neither did any of the above systems. Meaning this is an unexplored territory for testing and is highly relevant as the socialness greatly affects the availability of the video pieces over time.
%qiu and serkant, not mentioned probably do that


%%% Local Variables:
%%% mode: latex
%%% TeX-master: "../ClassicThesis"
%%% End:
